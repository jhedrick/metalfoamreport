\documentclass[compileTAMUreport.tex]{subfiles}

\begin{document}

The usage of metal foams in the boiler to enhance the heat transfer between the boiler and fluid  would effectively increase the rate at which the steam is produced thereby potentially reducing the length of the boiler exponentially. % I DONT by the exponential part? Proof?
The analysis of heat transfer in a metal foam can be done in two energy equation volume averaged and cell/lattice resolution method.

In the volume averaged method the effective surface area, relative density and effective heat transfer coefficient  are used to calculate the various heat transfer properties required to determine the desired results. 
%Cite this\{Du 2010}

In the cell resolution method, the unit cell of a metal foam in the particular condition to be analyzed is designed based on various assumption made, the cell design is used to run in various simulation softwares in various desired condition.
%Cite this\{Kopanidis2010 }

In either method, the fluids equations are the same, and do not change:
Solving the conservation of momentum first, before invoking the heat transfer solver, will allow the "fully developed" flow condition to be realized.


Note that for non-thermal equilibrium, the temperature within the foam is not assumed to be equal to the wall fluid boundary temperature. This is of great concern when you have a fluids with a much lower thermal conductivity than the metal foam solid. 

Lastly, you can use the following relations for normal and shear stress with velocity gradients
The bound on heat flux into the tubes are given the the Zuber Kutateladze prediction for film boiling, such that 


Since the basic aim of this project is to compare the effectiveness of a metal foam when used inside a boiler as well as to find the sensitivity of the heat transfer based on the various properties of a metal foam. Based on the previous work done in this field, the following properties have been taken into consideration for the simulation and analysis of the model

For the simulation of our model we have taken the ligament for the metal foam to be triangular and the overall shape of the cell to be tetrakiadecaheadron (14 phases - 8 equilateral triangle and 6 octagons).

\begin{equation}
1=1
\end{equation}


\end{document}
\documentclass[compileTAMUreport.tex]{subfiles}

\begin{document}

\chapter{Introduction}
Metal foams have been an increasing area of interest to scientists and engineers, especially in the field of thermodynamics and heat transfer. 
The foam provides a network of extended surfaces which increases the overall surface area for the dissipation of energy. 
Open cell metal foams have profound applications in heat transfer since these light-weight structures enable the creation of heat exchangers with heat transfer rates equivalent to or better than fin or plate type heat exchangers which are typically large and heavier.
\cite{Calmidi2000}
The amount of heat dissipation for a working fluid with metal foams is approximately 2-3 times more than the heat dissipation  without metal foam.
\cite{Ken2007}
Recent studies have shown that although the heat dissipation increases while using a metal foam, the cost of the metal foam is in the form a of high pressure drop that depends upon the foams permeability.
\cite{DuPlessis1994,Xianbing2012}
Under ideal conditions, metal foams can be used as a heat transfer medium thereby increase the overall efficiency of the system in many conditions.


After a thorough literature review, encompassing over 100 papers and other resources, literature relevant to two phase flow within a metal foam was limited and self described as inaccurate, thus the need for performance quantification should prove to be highly beneficial to increasing power generation efficiency.
\cite{Zhao2009}
The goal of this paper is to improve upon the work of Du in 2010 that analytically studied this exact problem, Kopanidis et al. in 2010 that numerically studied metal foams using a conjugate heat transfer (CHT) model, and lastly, Calmidi \& Mahajan in 2000 that studied forced convection through metal foam blocks. 
\cite{Du2010,Kopanidis2010,Calmidi2000}

This work adds to their contribution by quantifying the heat transfer enhancement within boiler tubes in the nucleate boiling regime under the assumption of non-thermal equilibrium by using metal foams of pore density. Porosity $\epsilon$ is also an independent variable, but seeking to minimize pressure losses in the boiler which is directly related to permeability $K$ which is a function of $\epsilon$, this work maximizes $\epsilon = 0.97$ as produced by ERG Corp.'s Duocell foam.
%This will be achieved by comparing the derived analytical solutions, derived using a novel method for analytical symmetry, to computational solutions performed with Star-CCM+ multiphysics CHT code. 
Through an analytical decomposition, this complicated problem can be simplified to a crude but still significantly refined compared to alternative methods such as volume averaging techniques which do not work well with turbulent flows. % From the analytical solutions, the sensitivity of the heat transfer enhancement to the variation in parameters will also be determined. 
%%% I don't think we need to intro the sensitivity here...

\end{document}
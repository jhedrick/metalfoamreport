\documentclass[asme2ejs.tex]{subfiles}

\begin{document}


%%%%%%%%%%%%%%%%%%%%%%%%%%%%%%%%%%%%%%%%%%%%%%%%%%%%%%%%%%%%%%%%%%%%%%
%\begin{abstract}
%{\it 
%This article illustrates preparation of the final version of
%an ASME journal paper submitted for publication using 
%\LaTeX2\raisebox{-.3ex}{$\epsilon$}. For the convenience of proofreading
%and editing, the final version shall be formatted in a single column. 
%This article is formatted based on the contents in the article
%entitled ``{\rm An ASME Journal Article Created Using 
%\LaTeX2\raisebox{-.3ex}{$\epsilon$} in ASME Format for 
%Testing Your Figures,}" which is a template for 
%preparation of ASME papers submitted for review.
%An abstract for an ASME paper should be less than 150 words and is normally in italics.  Notice that this abstract is to be set in 9pt Times Italic, single spaced and right justified.  
%%%% 
%Please use this template to test how your figures will look on the printed journal page of the Journal of Mechanical Design.  The Journal will no longer publish papers that contain errors in figure resolution.  These usually consist of unreadable or fuzzy text, and pixilation or rasterization of lines.  This template identifies the specifications used by JMD some of which may not be easily duplicated; for example, ASME actually uses Helvetica Condensed Bold, but this is not generally available so for the purpose of this exercise Helvetica is adequate.  However, reproduction of the journal page is not the goal, instead this exercise is to verify the quality of your figures. 
%}
%\end{abstract}


\end{document}
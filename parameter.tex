\documentclass[compileTAMUreport.tex]{subfiles}

\begin{document}


\section{Parameter�s taken in to consideration} 
\begin{itemize}

\item Pores per inch ($\mathrm{PPI}$): As the name itself says, it used to describe the number of pores one cubic inch of the metal foam. This when altered will effect size and flow the fluid in the foam, which would affect the overall heat transfer to fluid from the foam.\cite{Miwa2012}

\item Porosity ($\epsilon$): porosity is used to indicate the fraction of the empty space in a cell of the foam. This along with the Pores per inch will the overall size of the cell, which will affect the heat transfer from the foam.\cite{Zhao2004}

\item Velocity of the fluid:  the velocity of the fluid will affect the Reynolds number which will affect the Nusselt number and the convective heat transfer coefficient.\cite{Dukhan2010}

\item Inlet Pressure: The pressure at the inlet would affect the saturation temperature at which the phase would take place and the amount of heat that has to be supplied to the boiler.
Inlet temperature: The inlet temperature of the fluid will affect the amount of heat required to get the fluid to saturation temperature. 

\item Diameter of the boiler pipes: The inlet velocity of the steam depends on the total amount of the steam required and the diameter of the pipes and the total number of the pipes present.

\item Number of Boiler pipes: As mentioned above the number of the pipes present, diameter of the pipes, and the overall amount of the steam required will affect the inlet velocity.

\item Flow regime: The flow regime inside the boiler pipes basically defines the various arrangements for the gas and liquid phase. Depending on the alignment of the pipe and relative velocity of the gas phase with respect to the liquid the flow regime is considered. \cite{Yang2009}

\item Cell Structure: The cell structure is a very important factor as it defines the size of the ligaments and the nodes. Which will affect the flow and also define the various factors to be considered at the cellular and nodal level.\cite{Boomsma2011}

\item Shape of the filament: This is the factor in cellular level analysis. As the shape of the filament will affect the heat transfer between the filament and fluid as well as the heat transfer between the two nodes. Which will affect the temperature profile of the metal foam in the radial direction. \cite{JAEGER2013} 

\item Shape of the node:  The shape of the node will affect the convective heat transfer between the filament and node and the heat transfer between the node and the fluid.

\item Outlet Quality of the steam: The outlet quality of the steam will affect the length of the pipe. The quality of the steam also define the amount power that can be extracted from the steam if the saturated temperature and pressure is defined  

\item Length of the pipe: The length of the pipe will affect the quality of the steam and the Reynolds number for the convective heat transfer between the pipe and the fluid.

\item Angle of inclination of the pipe: The angle of inclination of the boiler pipe is going to effect the flow regime and the steam quality. As the effect of gravity is going to come into play throught the flow of the pipe in the  direction of flow, but when the pipe is horizontal gravity will affect in the type of the flow regime being formed and position of each of the fluid in the regime.\cite{Yang2009}


\item Permeability: It is the ability of the metal foam to transmit the fluids through the pores. It effects the pressure drop in the metal foam.

\item Boundary Conditions: The boundary conditions vary the temperature distribution along the radial direction of the foam. 

\end{itemize}

\end{document}

